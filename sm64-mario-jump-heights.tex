\documentclass[11pt]{article}
\usepackage[margin=2cm]{geometry}
\usepackage{hyperref}
\usepackage{amsmath}
\usepackage{amsfonts}
\usepackage{amsthm}
\usepackage{xcolor}

\setlength \parindent{0pt}

\title{Super Mario 64 - Mario's Jump Height}
\date{December 23rd, 2025}
\author{Chase Bradley}

\begin{document}

\maketitle

\newpage
\tableofcontents

\newpage
\section{Calculating Jump Height}
\label{sec:height-function}

To find how big any given jump goes, we could spend the next week trying various
jumps in-game and seeing how high they go, but that sucks and ain't nobody got
time for that.  Instead, we're going to derive a function which, given an
initial vertical velocity and a gravity constant, can find the peak heigh of
Mario's jump.

\subsection{Velocity Function}
\label{sec:height-function-velocity}

Let $v_0 \in \mathbb{R}^{+}$ be the velocity of Mario right after entering an
airborne state.  We assume a positive or zero velocity for now to simplify our
height function.\\

Let $g \in (\mathbb{R}^{+} - \{0\})$ be a gravity constant.  This should remain constant
throughout the jump, but it can change based on action, such as long jumps.\\

We'll define a function $v_g(t): \mathbb{R}^{+} \to \mathbb{R}$ which represents
Mario's velocity after $t$ in-game ticks since entering an airborne state as the
following:

\begin{eqnarray*}
v_g(t) &=& max(v_0 - g \lfloor t \rfloor, -75) \\
\end{eqnarray*}

We take the max because Mario's terminal velocity is 75 units downward per
frame.  We also take the floor of $t$ so we have a domain over
$\mathbb{R}^{+}$ instead of over $\mathbb{Z}^{+}$.  This will be useful for
deriving the position function.\\

\subsection{Position Function}
\label{sec:height-function-position}

Next, we're going to derive the position function,
$r_g(t): \mathbb{R}^{+} \to \mathbb{R}$ which calculates Mario's vertical
displacement from his original jump location after $t$ in-game ticks with a
gravity of $g$.\\

To solve this, we will use calculus, but with some minor liberties.  Since we're
using the floor function and the max function, our function is not easy to
integrate.  In particular, the floor function means we can't use regular
integration to solve this because the function isn't continuous.  In addition,
the max function secretly turns our function into a piecewise function.  To
solve both these problems, we're going to do the following:

\begin{itemize}
\item Define a continuous function which represents the antiderivative of the floor function
\item Make $v_g(t)$ piecewise so we can integrate each component seperately
\end{itemize}

For the first point, we will define the following antiderivative of the floor
function:

\begin{eqnarray*}
\int_0^x \lfloor t \rfloor dt &=& \frac{1}{2} \lfloor x \rfloor (2x - 1 - \lfloor x \rfloor) \\
\end{eqnarray*}

This function has the nice property that it's continuous, as well as having a
linear slope which more accurately interpolates the non-integer values.  This
also ensures the function is monotonic for $x \geq 0$.  To explain why both
these points are important, consider the following alternative antiderivative:

\begin{eqnarray*}
\int_0^x \lfloor t \rfloor dt &=& \frac{x(x-1)}{2} \\
\end{eqnarray*}

This function works fine for positive integer values, but it begins to show
issues with non-integer values.  On the interval $[0,1]$, our function actually
goes negative.  This doesn't make any sense, because we're accumulating strictly
positive quantities.  In addition, quadratic interpolation doesn't make sense
because each interval of the floor function is analygous to integrating a
constant, which should be linear.  Thus, we'd expect linear interpolation
between each integer point, not quadratic interpolation.  Thus, this function
isn't suitable.\\

Next, we'll redefine $v_g(t)$ to be piecewise, free of the max function.  This
will allow us to integrate each component of the function independently, then
combine together as a new piecewise function after integration.

\begin{eqnarray*}
v_g(t) &=& \begin{cases}
   v_0 - g \lfloor t \rfloor & v_0 - g \lfloor t \rfloor > -75 \\
   -75 & v_0 - g \lfloor t \rfloor \leq -75 \\
\end{cases}
\end{eqnarray*}

With these restrictions lifted, we can now compute $r_g(t)$.  Note that for the
terminal velocity, we'll use a generic lower integral bound $\alpha$.  This is
because it's tricky to derive this, and will later turn out to be irrelevant.

\begin{eqnarray*}
r_g(t) &=& \int_0^t v_g(w) dw \\
&=& \begin{cases}
   \int_0^t (v_0 - g \lfloor w \rfloor) dw & v_0 - g \lfloor t \rfloor > -75 \\
   \int_{\alpha}^t (-75) dw & v_0 - g \lfloor t \rfloor \leq -75 \\
\end{cases} \\
&=& \begin{cases}
   v_0 t - g \int_0^t \lfloor w \rfloor dw & v_0 - g \lfloor t \rfloor > -75 \\
   -75(t - \alpha) & v_0 - g \lfloor t \rfloor \leq -75 \\
\end{cases} \\
&=& \begin{cases}
   v_0 t - \frac{g}{2} \lfloor t \rfloor (2t - 1 - \lfloor t \rfloor) & v_0 - g \lfloor t \rfloor > -75 \\
   -75(t - \alpha) & v_0 - g \lfloor t \rfloor \leq -75 \\
\end{cases} \\
\end{eqnarray*}

With this, we now have derived the following position function:

\begin{eqnarray*}
r_g(t) &=& \begin{cases}
   v_0 t - \frac{g}{2} \lfloor t \rfloor (2t - 1 - \lfloor t \rfloor) & v_0 - g \lfloor t \rfloor > -75 \\
   -75(t - \alpha) & v_0 - g \lfloor t \rfloor \leq -75 \\
\end{cases}
\end{eqnarray*}

However, this is not entirely accurate.  Since the game runs in discrete ticks,
we should also have the position function be discrete.  This allows us to make
some simplifications.

\begin{eqnarray*}
r_g(t) &=& \begin{cases}
   v_0 t - \frac{g}{2} \lfloor t \rfloor (2t - 1 - \lfloor t \rfloor) & v_0 - g \lfloor t \rfloor > -75 \\
   -75(t - \alpha) & v_0 - g \lfloor t \rfloor \leq -75 \\
\end{cases} \\
&=& \begin{cases}
   v_0 \lfloor t \rfloor - \frac{g}{2} \lfloor t \rfloor (2 \lfloor t \rfloor - 1 - \lfloor t \rfloor) & v_0 - g \lfloor t \rfloor > -75 \\
   -75(\lfloor t \rfloor - \alpha) & v_0 - g \lfloor t \rfloor \leq -75 \\
\end{cases} \\
&=& \begin{cases}
   \lfloor t \rfloor (v_0 - \frac{g}{2} (\lfloor t \rfloor - 1)) & v_0 - g \lfloor t \rfloor > -75 \\
   -75(\lfloor t \rfloor - \alpha) & v_0 - g \lfloor t \rfloor \leq -75 \\
\end{cases} \\
\end{eqnarray*}

This gives us our final position function:

\begin{eqnarray*}
r_g(t) &=& \begin{cases}
   \lfloor t \rfloor (v_0 - \frac{g}{2} (\lfloor t \rfloor - 1)) & v_0 - g \lfloor t \rfloor > -75 \\
   -75(\lfloor t \rfloor - \alpha) & v_0 - g \lfloor t \rfloor \leq -75 \\
\end{cases} \\
\end{eqnarray*}

For the curious reader, the following is what $\alpha$ should be:

\begin{eqnarray*}
\alpha &=& \lceil \frac{75 + v_0}{g} \rceil + \frac{1}{75} \int_{0}^{\lceil \frac{75 + v_0}{g} \rceil} (v_0 - g \lfloor t \rfloor) dt \\
\end{eqnarray*}

Derivation of this constant is left as an excercise.

\subsection{Height Function}
\label{sec:height-function-height}

We will now find a function $h_g(v): \mathbb{R}^{+} \to \mathbb{R}^{+}$
which computes the maximum value of the position function, given the starting
velocity $v$ and constant gravity $g$.\\

We will first makes some simplifications to $v_g(t)$ and $r_g(t)$.  Since we are
concerned with the maximum values, we aren't interested in negative velocities.
Thus, we will ignore terminal velocity in our modified velocity and position
functions.

\begin{eqnarray*}
v_g(t) &=& v_0 - g \lfloor t \rfloor \\
r_g(t) &=& \lfloor t \rfloor (v_0 - \frac{g}{2} (\lfloor t \rfloor - 1)) \\
\end{eqnarray*}

Next, we want to find the tick which has the maximum height.  We can accomplish
this by finding when the velocity is zero.

\begin{eqnarray*}
v_g(t) &=& 0 \\
v_0 - g \lfloor t \rfloor &=& 0 \\
- g \lfloor t \rfloor &=& -v_0 \\
g \lfloor t \rfloor &=& v_0 \\
\lfloor t \rfloor &=& \frac{v_0}{g} \\
\end{eqnarray*}

Here, we will make another technical decision.  Since the floor function is not
surjective, it may not be possible to find a value for $t$ which satisfies this
equation without the floor function.\\

To solve this, we're going to choose to use the ceil function.  This must be
used since we may have no individual frame where $v_g(t) = 0$.  However, the
first tick where $v_g(t) < 0$ will always maximize $r_g(t)$.  Thus, we'll get
the following value for $t$:

\begin{eqnarray*}
t &=& \lceil \frac{v_0}{g} \rceil \\
\end{eqnarray*}

We can now plug this in to $r_g(t)$ to get our height function $h_g(v)$.

\begin{eqnarray*}
h_g(v) &=& r_g(\lceil \frac{v}{g} \rceil) \\
&=& \lfloor \lceil \frac{v}{g} \rceil \rfloor (v - \frac{g}{2} (\lfloor \lceil \frac{v}{g} \rceil \rfloor - 1)) \\
&=& \lceil \frac{v}{g} \rceil (v - \frac{g}{2} (\lceil \frac{v}{g} \rceil - 1)) \\
\end{eqnarray*}

This cannot be simplified further.  Given $x,y \in \mathbb{R}$, $y \lceil \frac{x}{y} \rceil = \lceil x \rceil$
is not generally true.  Thus, our final height function is the following:

\begin{eqnarray*}
h_g(v) &=& \lceil \frac{v}{g} \rceil (v - \frac{g}{2} (\lceil \frac{v}{g} \rceil - 1)) \\
\end{eqnarray*}

\subsection{Inverse Height Function}
\label{sec:height-function-inverse-height}

Lastly, we will find an inverse for the height function
$h^{-1}_g (r): \mathbb{R}^{+} \to \mathbb{R}^{+}$. This will be useful when we
want to find the required velocity to travel a given height with a given gravity
constant.\\

We're also going to make the following modification and define $v$ in terms of
an arbitrary function.  This will be invaluable later when we want to find
$h^{-1}_g$ with initial velocity as a function.  For example, initial vertical
velocity can be a function of forward running speed.  Thus, we'll define the
following functions:\\

\begin{eqnarray*}
v &:& A \to \mathbb{R}^{+} \text{ such that } A \subseteq \mathbb{R}, 0 \in \text{\textsf{Im}}(v), v \text{ is injective } \\
v^{-1} &:& \mathbb{R}^{+} \to A \text{ such that for } \forall a \in A, v^{-1}(v(a)) = a \\
\end{eqnarray*}

We will now consider given $r \in \mathbb{R}^{+}$,
$g \in (\mathbb{R}^{+} - \{0\})$, the function $h_g(v(a))$ attempt to find
$a \in A$ such that $h_g(v(a)) = r$.  Rearranged, we are trying to find
$a \in A$ such that $a = v^{-1}(h^{-1}_g(r))$.\\

To solve this, we're going to show that $h_g(v(a))$ can be written as a
piecewise function such that for certain intervals of $h_g$, it can be written
in the form of $mv(a) + b$ such that $m, b \in \mathbb{R}$. We will then use
this fact to find $m$ and $b$ when given $r$, then finally solve for $a$.\\

First, we're going to find all the intervals where $h_g(v(a))$ can be written as
$mv(a) + b$.\\

\textit{Proposition}.  Given $g \in (\mathbb{R}^{+} - \{0\})$ and
$n \in \mathbb{Z}^{+}$, then
$\exists m_n, b_n \in \mathbb{R}: h_g(v(a)) = m_n v(a) + b_n$ on the interval
$(v^{-1}(gn), v^{-1}(g(n + 1))]$.

\begin{proof}

Recall that for $\forall n \in \mathbb{Z}^{+}$,
$\lim_{\alpha \to 0^{+}} \lceil n + \alpha \rceil = n + 1$.\\

Now look at our function
$h_g(v(a)) = \lceil \frac{v(a)}{g} \rceil (v(a) - \frac{g}{2}(\lceil \frac{v(a)}{g} \rceil - 1))$.
In particular, focus on the $\lceil \frac{v(a)}{g} \rceil$ term.  If we assume
$\lceil \frac{v(a)}{g} \rceil$ to be constant $C$, then we get the following:

\begin{eqnarray*}
h_g(v(a)) &=& \lceil \frac{v(a)}{g} \rceil (v(a) - \frac{g}{2}(\lceil \frac{v(a)}{g} \rceil - 1)) \\
&=& C(v(a) - \frac{g}{2}(C - 1)) \\
&=& Cv(a) - C \frac{g}{2}(C - 1) \\
&=& Cv(a) - \frac{gC(C - 1)}{2} \\
&=& Cv(a) + (\frac{-gC(C - 1)}{2}) \\
\end{eqnarray*}

This is in the form of $h_g(v(a)) = m_n v(a) + b_n$.  Therefore, if
$\lceil \frac{v(a)}{g} \rceil$ is constant over the interval $I$, then
$\exists m,b \in \mathbb{R}: h_g(v(a)) = mv(a) + b$ over the interval $I$.\\

Next, we're going to show that $\lceil \frac{v(a)}{g} \rceil$ remains constant
over the interval $(v^{-1}(gn), v^{-1}(g(n + 1))]$.\\

To do this, we're going to show that
$\lim_{\alpha \to 0^{+}} \lceil \frac{v(v^{-1}(gn))}{g} + \alpha \rceil = \lceil \frac{v(v^{-1}(g(n + 1)))}{g} \rceil$.\\

\begin{eqnarray*}
\lim_{\alpha \to 0^{+}} \lceil \frac{v(v^{-1}(gn))}{g} + \alpha \rceil &=& \lim_{\alpha \to 0^{+}} \lceil \frac{gn}{g} + \alpha \rceil \\
&=& \lim_{\alpha \to 0^{+}} \lceil n + \alpha \rceil \\
&=& n + 1 \\
\end{eqnarray*}

\begin{eqnarray*}
\lceil \frac{v(v^{-1}(g(n + 1)))}{g} \rceil &=& \lceil \frac{g(n + 1))}{g} \rceil \\
&=& \lceil n + 1 \rceil \\
&=& n + 1 \\
\end{eqnarray*}

Since
$\lim_{\alpha \to 0^{+}} \lceil \frac{v(v^{-1}(gn))}{g} + \alpha \rceil = \lceil \frac{v(v^{-1}(g(n + 1)))}{g} \rceil$,
we know that $\lceil \frac{v(a)}{g} \rceil$ is constant over the interval
$(v^{-1}(gn), v^{-1}(g(n + 1))]$.\\

As a result, this means that
$\exists m_n, b_n \in \mathbb{R}: h_g(v(a)) = m_n v(a) + b_n$ over the interval\\
$(v^{-1}(gn), v^{-1}(g(n + 1))]$.

\end{proof}

With this knowledge, we now know if we can find $n$ in terms of $r$, we can then
rewrite $h_g (v(a)) = m_n v(a) + b_n$ over the interval
$(v^{-1}(gn), v^{-1}(g(n + 1))]$.  Assuming we already have $m_n$ and $b_n$,
this is trivial to solve.

\begin{eqnarray*}
h_g(v(a)) &=& r \\
m_n v(a) + b_n &=& r \\
m_n v(a) &=& r - b_n \\
v(a) &=& \frac{r - b_n}{m_n} \\
h^{-1}_g(r) &=& \frac{r - b_n}{m_n} \\
v^{-1}(h^{-1}_g(r)) &=& v^{-1}(\frac{r - b_n}{m_n}) \\
\end{eqnarray*}

Next, we're going to solve for $n$ in terms of $r$ such that the interval for
$n$ contains our solution.  To do this, we're going to plug in the start point
of each interval and then try to solve for $n$ in terms of $r$.

\begin{eqnarray*}
h_g(v(v^{-1}(gn))) &=& r \\
h_g(gn) &=& r \\
\lceil \frac{gn}{g} \rceil (gn - \frac{g}{2}(\lceil \frac{gn}{g} \rceil - 1)) &=& r \\
\lceil n \rceil (gn - \frac{g}{2}(\lceil n \rceil - 1)) &=& r \\
n (gn - \frac{g}{2}(n - 1)) &=& r \\
gn (n - \frac{1}{2}(n - 1)) &=& r \\
\frac{gn}{2} (2n - (n - 1)) &=& r \\
\frac{gn}{2} (2n - n + 1) &=& r \\
\frac{gn}{2} (n + 1) &=& r \\
n(n + 1) &=& \frac{2r}{g} \\
n^2 + n &=& \frac{2r}{g} \\
n^2 + n - \frac{2r}{g} &=& 0 \\
(1)n^2 + (1)n + (\frac{-2r}{g}) &=& 0 \\
n &=& \frac{-(1) + \sqrt{(1)^2 - 4(1)(\frac{-2r}{g})}}{2(1)} \\
n &=& \frac{-1 + \sqrt{1 + \frac{8r}{g}}}{2} \\
n &=& \frac{-1}{2} + \frac{1}{2}\sqrt{1 + \frac{8r}{g}} \\
n &=& \frac{1}{2}\sqrt{1 + \frac{8r}{g}} - \frac{1}{2} \\
n &=& \sqrt{\frac{1}{4}(1 + \frac{8r}{g})} - \frac{1}{2} \\
n &=& \sqrt{\frac{1}{4} + \frac{8r}{4g}} - \frac{1}{2} \\
n &=& \sqrt{\frac{1}{4} + \frac{2r}{g}} - \frac{1}{2} \\
n &=& \sqrt{\frac{2r}{g} + \frac{1}{4}} - \frac{1}{2} \\
\end{eqnarray*}

This will give us a real number which is on the interval $[n, n + 1)$.  We get
the exact value for $n$ by taking the floor of the above.  This results in our
final solution for $n$:

\begin{eqnarray*}
n &=& \lfloor \sqrt{\frac{2r}{g} + \frac{1}{4}} - \frac{1}{2} \rfloor \\
\end{eqnarray*}

We will now solve for $m_n$ and $b_n$ in terms of $n$.  This is actually pretty
simple.  We just plug in for the endpoint of the $n$'th interval for all the
ceil terms and write it in the correct form.\\

To help clean things up, we're going to first simplify
$\lceil \frac{v(v^{-1}(g(n + 1)))}{g} \rceil$.

\begin{eqnarray*}
\lceil \frac{v(v^{-1}(g(n + 1)))}{g} \rceil &=& \lceil \frac{g(n + 1)}{g} \rceil \\
&=& \lceil n + 1 \rceil \\
&=& n + 1 \\
\end{eqnarray*}

Let's now plug in to $h_g(v(a))$ on the $n$'th interval.

\begin{eqnarray*}
h_g(v(a)) &=& \lceil \frac{v(a)}{g} \rceil (v(a) - \frac{g}{2}(\lceil \frac{v(a)}{g} \rceil - 1)) \\
&=& \lceil \frac{v(v^{-1}(g(n + 1)))}{g} \rceil (v(a) - \frac{g}{2}(\lceil \frac{v(v^{-1}(g(n + 1)))}{g} \rceil - 1)) \\
&=& (n + 1) (v(a) - \frac{g}{2}((n + 1) - 1)) \\
&=& (n + 1) (v(a) - \frac{gn}{2}) \\
&=& (n + 1)v(a) - \frac{gn(n + 1)}{2} \\
&=& (n + 1)v(a) + \frac{-gn(n + 1)}{2} \\
h_g(v(a)) &=& (n + 1)v(a) + \frac{-gn(n + 1)}{2} \\
\end{eqnarray*}

With this, we now have $m_n$ and $b_n$ in terms of $n$.

\begin{eqnarray*}
m_n &=& n + 1 \\
b_n &=& \frac{-gn(n + 1)}{2} \\
\end{eqnarray*}

Now, let's revisit our previous solution for $v^{-1}(h^{-1}_g(r))$ and plug in
for $m_n$ and $b_n$.\\

\begin{eqnarray*}
v^{-1}(h^{-1}_g(r)) &=& v^{-1}(\frac{r - b_n}{m_n}) \\
&=& v^{-1}(\frac{r - \frac{-gn(n + 1)}{2}}{n + 1}) \\
&=& v^{-1}(\frac{r + \frac{gn(n + 1)}{2}}{n + 1}) \\
&=& v^{-1}(\frac{r}{n + 1} + \frac{\frac{gn(n + 1)}{2}}{n + 1}) \\
&=& v^{-1}(\frac{r}{n + 1} + \frac{gn}{2}) \\
v^{-1}(h^{-1}_g(r)) &=& v^{-1}(\frac{r}{n + 1} + \frac{gn}{2}) \\
\end{eqnarray*}

Finally, we write $n$ in terms of $r$ to get our final solution to $v^{-1}(h^{-1}_g(r))$.\\

We could just plug in for $n$, but that would result in a butt ugly equation
with no way to simplify.  Instead, we're going to define
$n_g(r): \mathbb{R}^{+} \to \mathbb{Z}^{+}$ to be our solution for $n$
given $r$, then use this function inside our definition for $h_g^{-1}(r)$.
This gives us our final answer for $h_g^{-1}(r)$.

\begin{eqnarray*}
n_g(r) &=& \lfloor \sqrt{\frac{2r}{g} + \frac{1}{4}} - \frac{1}{2}\rfloor \\
h_g^{-1}(r) &=& \frac{r}{1 + n_g(r)} + \frac{g n_g(r)}{2} \\
\end{eqnarray*}

Notice that there is no dependence on $v^{-1}$ to compute $h_g^{-1}$.
Therefore, it's redundant to specify $v^{-1} \circ h_g^{-1}$, and we can treat $v$
as if it's a regular variable in terms of our solution.\\

\newpage
\subsection{Function Index}
\label{sec:height-function-index}

For reference, here is a table of all relevant functions for this section:

\begin{center}
\begin{tabular}{|c|c|c|c|}
\hline
Name & Identifier & Domain and Codomain & Definition \\
\hline
Velocity Function & $v_g(t)$ & $\mathbb{R}^{+} \to \mathbb{R}$ & $v_g(t) = max(v_0 - g \lfloor t \rfloor, -75)$ \\
\hline
Position Function & $r_g(t)$ & $\mathbb{R}^{+} \to \mathbb{R}$ & $r_g(t) = \begin{cases}
   \lfloor t \rfloor (v_0 - \frac{g}{2} (\lfloor t \rfloor - 1)) & v_0 - g \lfloor t \rfloor > -75 \\
   -75 (\lfloor t \rfloor - \alpha) & v_0 - g \lfloor t \rfloor \leq -75 \\
\end{cases}$ \\
\hline
Height Function & $h_g(v)$ & $\mathbb{R}^{+} \to \mathbb{R}^{+}$ & $h_g(v) = \lceil \frac{v}{g} \rceil (v - \frac{g}{2}(\lceil \frac{v}{g} \rceil - 1))$ \\
\hline
Inverse Height Interval & $n_g(r)$ & $\mathbb{R}^{+} \to \mathbb{Z}^{+}$ & $\lfloor \sqrt{\frac{2r}{g} + \frac{1}{4}} - \frac{1}{2} \rfloor$ \\
\hline
Inverse Height Function & $h_g^{-1}(r)$ & $\mathbb{R}^{+} \to \mathbb{R}^{+}$ & $h_g^{-1}(r) = \frac{r}{1 + n_g(r)} + \frac{g n_g(r)}{2}$ \\
\hline
\end{tabular}
\end{center}

\newpage
\section{Jumping Actions}
\label{sec:jumping-actions}

We will now analyze how different "jumping" actions set Mario's starting
vertical velocity and affect gravity, which can be used with the previously
derived height function to compute the peak height for each jumping action.

\subsection{Jumping Velocities}
\label{sec:jumping-actions-velocities}

We will classify a "jumping" action as a Mario action with the following
criteria:

\begin{itemize}
\item Either the \textsf{ACT\_GROUP\_AIRBORNE} or \textsf{ACT\_GROUP\_SUBMERGED} flag is set
\item Vertical velocity is set to a positive value just before entering the action
\item Vertical velocity is only affected by gravity while in the action
\item Mario can be purposefully controlled with the analog stick throughout the jump
\end{itemize}

Some actions set initial vertical velocity based on Mario's forward speed
(\textsf{m->forwardVel}).  Mario's forward speed will be denoted with
'$f$'.  In addition, some use Mario's vertical speed.  This will be denoted with
'$v$'.\\

In addition, some actions use \textsf{set\_mario\_y\_vel\_based\_on\_fspeed()}
instead of directly setting \textsf{m->vel[1]}.  These actions check if Mario is
either in quicksand (\textsf{m->quicksandDepth} $\geq 1$) or squished
(\textsf{m->squishTimer} $\neq 0$).  If he is, the intended initial vertical
velocity is halved.  These actions have their vertical velocity colored blue.
The given initial vertical velocity assumes Mario is neither in quicksand or
squished.\\

This leaves us with the following table of actions:

\begin{center}
\begin{tiny}
\begin{tabular}{|c|c|c|c|}
\hline
Name & Identifier & Initial Vertical Velocity (units/tick) & Gravity (units/tick$^2$)\\
\hline
Single Jump & \textsf{ACT\_SINGLE\_JUMP} & \color{blue} $42 + \frac{f}{4}$ & $4$ \\
Double Jump & \textsf{ACT\_DOUBLE\_JUMP} & \color{blue} $52 + \frac{f}{4}$ & $4$ \\
Triple Jump & \textsf{ACT\_TRIPLE\_JUMP} & \color{blue} $69$ & $4$ \\
Flying Triple Jump & \textsf{ACT\_FLYING\_TRIPLE\_JUMP} & \color{blue} $82$ & $4$ \\
Special Triple Jump (initial jump) & \textsf{ACT\_SPECIAL\_TRIPLE\_JUMP} & \color{blue} $69$ & $4$ \\
Special Triple Jump (landing bounce) & \textsf{ACT\_SPECIAL\_TRIPLE\_JUMP} & $42$ & $4$ \\
Jump With Object & \textsf{ACT\_HOLD\_JUMP} & \color{blue} $42 + \frac{f}{4}$ & $4$ \\
Jump From Poll (middle) & \textsf{ACT\_WALL\_KICK\_AIR} & \color{blue} $62$ & $4$ \\
Jump From Poll (top) & \textsf{ACT\_TOP\_OF\_POLL\_JUMP} & \color{blue} $62$ & $4$ \\
Jump While Riding Shell & \textsf{ACT\_RIDING\_SHELL\_JUMP} & $\color{blue} 42 + \frac{f}{4}$ & $4$ \\
Side Flip & \textsf{ACT\_SIDE\_FLIP} & \color{blue} $62$ & $4$ \\
Back Flip & \textsf{ACT\_BACKFLIP} & \color{blue} $62$ & $4$ \\
Dive (from ground) & \textsf{ACT\_DIVE} & $20$ & $4$ \\
Forward Rollout & \textsf{ACT\_FORWARD\_ROLLOUT} & $30$ & $4$ \\
Backward Rollout & \textsf{ACT\_BACKWARD\_ROLLOUT} & $30$ & $4$ \\
Sliding Kick (initial kick) & \textsf{ACT\_SLIDE\_KICK} & $12$ & $2$ \\
Sliding Kick (landing bounce) & \textsf{ACT\_SLIDE\_KICK} & $-\frac{v}{2}$ & $2$ \\
Jumping Kick & \textsf{ACT\_JUMP\_KICK} & $20$ & $4$ \\
Wall Kick & \textsf{ACT\_WALL\_KICK\_AIR} & \color{blue} $62$ & $4$ \\
Long Jump & \textsf{ACT\_LONG\_JUMP} & \color{blue} $30$ & $2$ \\
Lava Boost & \textsf{ACT\_LAVA\_BOOST} & $84$ & $3.2$ \\
Crazy Box Bounce (1st bounce) & \textsf{ACT\_CRAZY\_BOX\_BOUNCE} & $45$ & $4$ \\
Crazy Box Bounce (2nd bounce) & \textsf{ACT\_CRAZY\_BOX\_BOUNCE} & $60$ & $4$ \\
Crazy Box Bounce (3rd bounce) & \textsf{ACT\_CRAZY\_BOX\_BOUNCE} & $100$ & $4$ \\
Underwater Metal Cap Jump & \textsf{ACT\_METAL\_WATER\_JUMP} & $32$ & $1.6$ \\
Underwater Metal Cap Jump With Object & \textsf{ACT\_HOLD\_METAL\_WATER\_JUMP} & $32$ & $1.6$ \\
\hline
\end{tabular}
\end{tiny}
\end{center}

We will also include the following table of actions, which consist of exceptions
to the 4th criteria for a jumping action:

\begin{center}
\begin{tiny}
\begin{tabular}{|c|c|c|c|}
\hline
Name & Identifier & Initial Vertical Velocity (units/tick) & Gravity (units/tick$^2$)\\
\hline
Steep Jump & \textsf{ACT\_STEEP\_JUMP} & \color{blue} $42 + \frac{f}{4}$ & $4$ \\
Jump From Water & \textsf{ACT\_WATER\_JUMP} & \color{blue} $42$ & $4$ \\
Jump From Water With Object & \textsf{ACT\_HOLD\_WATER\_JUMP} & \color{blue} $42$ & $4$ \\
Jump While Burning & \textsf{ACT\_BURNING\_JUMP} & $31.5$ & $4$ \\
\hline
\end{tabular}
\end{tiny}
\end{center}

\subsection{Jumping Heights}
\label{sec:jumping-actions-heights}

We will use the height function $h_g(v)$ from
\hyperref[sec:height-function-height]{section 1.3} to calculate the peak jump
height for each jumping actions.\\

We will split jumping actions into two tables, based on whether their peak jump
height is constant or a function of either Mario's forward or vertical speed.\\

Actions which have their initial vertical velocity affected by either being
in quicksand or squished will have the peak jump height while affected also
provided.

\begin{center}
\begin{tiny}
\begin{tabular}{|c|c|c|c|}
\hline
Name & Identifier & Peak Jump Height (units) & Squished Peak Jump Height (units)\\
\hline
Triple Jump & \textsf{ACT\_TRIPLE\_JUMP} & $630$ & $166.5$ \\
Flying Triple Jump & \textsf{ACT\_FLYING\_TRIPLE\_JUMP} & $882$ & $231$ \\
Special Triple Jump (initial jump) & \textsf{ACT\_SPECIAL\_TRIPLE\_JUMP} & $630$ & $166.5$ \\
Special Triple Jump (landing bounce) & \textsf{ACT\_SPECIAL\_TRIPLE\_JUMP} & $242$ & \\
Jump From Poll (middle) & \textsf{ACT\_WALL\_KICK\_AIR} & $512$ & $136$ \\
Jump From Poll (top) & \textsf{ACT\_TOP\_OF\_POLL\_JUMP} & $512$ & $136$ \\
Side Flip & \textsf{ACT\_SIDE\_FLIP} & $512$ & $136$ \\
Back Flip & \textsf{ACT\_BACKFLIP} & $512$ & $136$ \\
Dive (from ground) & \textsf{ACT\_DIVE} & $60$ & \\
Forward Rollout & \textsf{ACT\_FORWARD\_ROLLOUT} & $128$ & \\
Backward Rollout & \textsf{ACT\_BACKWARD\_ROLLOUT} & $128$ & \\
Sliding Kick (initial kick) & \textsf{ACT\_SLIDE\_KICK} & $42$ & \\
Jumping Kick & \textsf{ACT\_JUMP\_KICK} & $60$ & \\
Wall Kick & \textsf{ACT\_WALL\_KICK\_AIR} & $512$ & $136$ \\
Long Jump & \textsf{ACT\_LONG\_JUMP} & $240$ & $64$ \\
Lava Boost & \textsf{ACT\_LAVA\_BOOST} & $1144.8$ & \\
Crazy Box Bounce (1st bounce) & \textsf{ACT\_CRAZY\_BOX\_BOUNCE} & $276$ & \\
Crazy Box Bounce (2nd bounce) & \textsf{ACT\_CRAZY\_BOX\_BOUNCE} & $480$ & \\
Crazy Box Bounce (3rd bounce) & \textsf{ACT\_CRAZY\_BOX\_BOUNCE} & $1300$ & \\
Underwater Metal Cap Jump & \textsf{ACT\_METAL\_WATER\_JUMP} & $336$ & \\
Underwater Metal Cap Jump With Object & \textsf{ACT\_HOLD\_METAL\_WATER\_JUMP} & $336$ & \\
Jump From Water & \textsf{ACT\_WATER\_JUMP} & $242$ & $66$ \\
Jump From Water With Object & \textsf{ACT\_HOLD\_WATER\_JUMP} & $242$ & $66$ \\
Jump While Burning & \textsf{ACT\_BURNING\_JUMP} & $140$ & \\
\hline
\end{tabular}
\end{tiny}
\end{center}

\begin{center}
\begin{tiny}
\begin{tabular}{|c|c|c|c|c|}
\hline
Name & Identifier & Peak Jump Height (units) & Squished Peak Jump Height (units) & Domain (units) \\
\hline
Single Jump & \textsf{ACT\_SINGLE\_JUMP} & $h_4(42 + \frac{f}{4})$ & $h_4(21 + \frac{f}{8})$ & $f \geq -184$ \\
Double Jump & \textsf{ACT\_DOUBLE\_JUMP} & $h_4(52 + \frac{f}{4})$ & $h_4(26 + \frac{f}{8})$ & $f \geq -224$ \\
Jump With Object & \textsf{ACT\_HOLD\_JUMP} & $h_4(42 + \frac{f}{4})$ & $h_4(21 + \frac{f}{8})$ & $f \geq -184$ \\
Jump While Riding Shell & \textsf{ACT\_RIDING\_SHELL\_JUMP} & $h_4(42 + \frac{f}{4})$ & $h_4(21 + \frac{f}{8})$ & $f \geq -184$ \\
Sliding Kick (landing bounce) & \textsf{ACT\_SLIDE\_KICK} & $h_2(- \frac{v}{2})$ & & $-75 \leq v \leq 0$\\
Steep Jump & \textsf{ACT\_STEEP\_JUMP} & $h_4(42 + \frac{f}{4})$ & $h_4(21 + \frac{f}{8})$ & $f \geq -184$ \\
\hline
\end{tabular}
\end{tiny}
\end{center}

Notice that we add a domain restriction for the non-constant jump heights.
This is because the height function requires the initial vertical velocity to be
either positive or zero.  In addition, it's impossible to have a sliding kick
landing bounce with a vertical velocity smaller than terminal velocity, which is
$-75$ for this action.\\

Also notice that we give variable jump height in terms of $h_g(v)$ without
plugging in and simplifying.  This is because there are not many simplifications
to be made, thus it's redundant to plug in for each function seperately.
Plugging in the velocity function and simplifying the height function is left as
an exercise for the reader.\\

We're now going to show the required speed to reach a jump height of
$r \in \mathbb{R}^{+}$ for all the variable jumping actions.  Like above, we
will give equations in terms of $h_g^{-1}(r)$ from
\hyperref[sec:height-function-inverse-height]{section 1.4} to avoid a nightmare
of horrendous equations which can't be simplified.  Also note that we don't
provide a function-specific domain here.  This is because it's redundant, as the
domain of $h_g^{-1}$ is always $\mathbb{R}^{+}$ by definition.\\

\begin{center}
\begin{tiny}
\begin{tabular}{|c|c|c|c|}
\hline
Name & Identifier & Required Speed (units/tick) & Squished Required Speed (units/tick) \\
\hline
Single Jump & \textsf{ACT\_SINGLE\_JUMP} & $4h_4^{-1}(r) - 168$ & $8h_4^{-1}(r) - 168$ \\
Double Jump & \textsf{ACT\_DOUBLE\_JUMP} & $4h_4^{-1}(r) - 208$ & $8h_4^{-1}(r) - 208$ \\
Jump With Object & \textsf{ACT\_HOLD\_JUMP} & $4h_4^{-1}(r) - 168$ & $8h_4^{-1}(r) - 168$ \\
Jump While Riding Shell & \textsf{ACT\_RIDING\_SHELL\_JUMP} & $4h_4^{-1}(r) - 168$ & $8h_4^{-1}(r) - 168$ \\
Sliding Kick (landing bounce) & \textsf{ACT\_SLIDE\_KICK} & $-2h_2^{-1}(r)$ & \\
Steep Jump & \textsf{ACT\_STEEP\_JUMP} & $4h_4^{-1}(r) - 168$ & $8h_4^{-1}(r) - 168$ \\
\hline
\end{tabular}
\end{tiny}
\end{center}

\subsection{Jumping Height Scales}
\label{sec:jumping-actions-scales}

A minor point, but something interesting to note.  There are two complexity
classes jumping action heights can be classified into.  Given height function
$h^j \in \{h_g : g \in \mathbb{R}^{+} - \{0\}\}$ for jumping action $j$ and
initial velocity $v \in \mathbb{R}^{+}$, $h^j$ can be classified into one of the
following complexity classes:

\begin{eqnarray*}
h^j(v) &\in& \Theta(1) \\
h^j(v) &\in& \Theta(v^2) \\
\end{eqnarray*}

This means the jump height is either always constant, or increases
proportionally with the square of the input.  This is especially interesting for
the second case, as this means as we increase the input, the jump height begins
to rapidly grow.\\

For example, if we perform a double jump with $f$ running speed, then a jump
with $2f$ running speed is proportional to $4$ times the height as the previous
jump.  With $3f$ speed, the jump height is proportional to $9$ times the height
as the previous jump.\\

All of the jumping actions with a constant peak height belong to $\Theta(1)$,
and all jumping actions which are a function of either Mario's forward speed or
vertical speed belong to $\Theta(v^2)$.

\newpage
\section{Applications}
\label{sec:applications}

This is all interesting, but how is this useful (other than being cool, of
course!)?\\

\subsection{Running Speed Required To Grab A Hangable Ceiling}
\label{sec:applications-hangable-ceiling}

Mario's vertical collision hitbox is 160 units tall.  That is, if a ceiling's
height is 160 units above Mario or more, he will be allowed to fit
underneath it.  In the case of grabbable ceilings, if Mario's next position is
less than 160 units below the ceiling's hitbox, he will grab the ceiling.
However, ceilings can only be grabbed from either a single jump or a double jump
state.\\

What if there is a ceiling we want to grab, where we know how high off the
ground is and we can build enough speed to grab it?  How fast do we need to be
moving for a double jump to reach high enough to grab the ceiling?\\

Most people would probably try some random movement, fail the jump, and give up.
More big brain people would hack the game to give mario some amount of speed
before a double jump, then try various speed values until one works.  However,
we can do better!  We can calculate the \textit{exact} speed required for such a
jump!\\

Let's take for example the hangable ceiling just above the main entrance to the
castle in castle grounds.  Let's try doing a double jump from the very edge of
either of the bridge's banisters.  If we look at the model data, the height of
the banister closest to the castle is $957$, and the height of the hangable
ceiling is $1692$.  Since Mario's vertical hitbox height is $160$ units, we need
to jump over $(1692 - 160) - 957 = 575$ units high from a double jump.\\

We solve this by plugging in $r = 575$ into the formula for double jumps from
\hyperref[sec:jumping-actions-heights]{the inverse height function table} and
only consider running speeds greater than that as solutions.

\begin{eqnarray*}
v^{-1}(h_g^{-1}(r)) &=& 4h_4^{-1}(575) - 208 \\
&=& 4(\frac{575}{1 + n_4(575)} + \frac{4 n_4(575)}{2}) - 208 \\
&=& 4(\frac{575}{1 + n_4(575)} + 2 n_4(575)) - 208 \\
\end{eqnarray*}

To be cleaner, we're going to simplify $n_4(575)$ independently then plug back
in once simplified.

\begin{eqnarray*}
n_g(r) &=& \lfloor \sqrt{\frac{2r}{g} + \frac{1}{4}} - \frac{1}{2} \rfloor \\
n_4(575) &=& \lfloor \sqrt{\frac{2(575)}{4} + \frac{1}{4}} - \frac{1}{2} \rfloor \\
&=& \lfloor \sqrt{\frac{575}{2} + \frac{2}{4}} - \frac{1}{2} \rfloor \\
&=& \lfloor \sqrt{\frac{575 + 2}{2}} - \frac{1}{2} \rfloor \\
&=& \lfloor \sqrt{\frac{577}{2}} - \frac{1}{2} \rfloor \\
\end{eqnarray*}

No pressure here, just use a calculator.

\begin{eqnarray*}
n_4(575) &=& 16
\end{eqnarray*}

We can now plug in to our original equation.

\begin{eqnarray*}
v^{-1}(h_g^{-1}(r)) &=& 4(\frac{575}{1 + n_4(575)} + 2 n_4(575)) - 208 \\
&=& 4(\frac{575}{1 + 16} + 2(16)) - 208 \\
&=& 4(\frac{575}{17} + 32) - 208 \\
&=& 4(\frac{575}{17} + \frac{544}{17}) - 208 \\
&=& 4(\frac{575 + 544}{17}) - 208 \\
&=& 4(\frac{1119}{17}) - 208 \\
&=& \frac{4476}{17} - 208 \\
&=& \frac{4476}{17} - \frac{3536}{17} \\
&=& \frac{4476 - 3536}{17} \\
&=& \frac{940}{17} \\
\end{eqnarray*}

Thus, we need over $\frac{940}{17}$ forward speed into a double jump to grab the
hangable ceiling just above the castle entrance from the close end of the
bridge's banister.  Written as a decimal approximation, this is
approximately $55.294118$ units per frame of forward speed.\\

If you experiment with this in-game, you may notice that there is some
inaccuracy.  For example, $55.293$ forward speed may work.  This is due to
floating-point rounding errors accumulating through the game's physics logic
over several frames.  Our solution assumes zero precision issues, which does not
reflect reality.  Thus, the exact floating-point bits will differ based on
version of the game, platform, processor, rounding modes, and compiler.  For
example, on my laptop running an Intel Core i7-1165G7 with nearest rounding and
testing with commit \textsf{d7ca2c0} of \textsf{sm64ex} compiled with
\textsf{gcc-15.2.1}, the exact minimum floating-point value for Mario's forward
speed is \textsf{0x425d2d05}, or approximately $55.293964$ units per tick.  This
is a difference of $-0.000154$ from the expected result.\\

Despite this, our method is still preferable because it removes a huge amount
of unreliable experimentation, and we can instead focus around testing the range
of floating-point error around our expected result.  We can also do many other
things, since we have Mario's peak jump height and required speed as
mathematical functions, not random numbers from experiments.

\subsection{Optimizing For Multiple Variables}
\label{sec:applications-optimization}

Imagine the case where we are jumping up a steep slope.  Our forward speed
decreases the higher we run up the slope, but our starting height increases
the higher we go up the slope.\\

If we are looking to maximize height in this circumstance, we could try jumping
every frame we're on the slope to maximize this value.  What we can do is set up
a function which represents our jump height after running a given distance up
the slope, then use Calculus to optimize this function.

\end{document}
